\documentclass[11pt,twoside]{article} % use larger type; default would be 10pt
\raggedbottom

% usual packages loading:
%\usepackage{luatextra}
\usepackage{graphicx} % support the \includegraphics command and options
\usepackage{geometry} % See geometry.pdf to learn the layout options. There are lots.
\usepackage{gregoriotex} % for gregorio score inclusion
\usepackage{fullpage} % to reduce the margins
\usepackage[toc,page]{appendix}

\pdfminorversion=4

% choose the language of the document here
\usepackage[latin,frenchb]{babel}

% use the two following package for using normal TeX fonts
\usepackage[T1]{fontenc}
\usepackage[utf8]{luainputenc}

% If you use usual TeX fonts, here is a starting point:
\usepackage{times}

\usepackage{layout}

\usepackage{parallel}
\usepackage{color}
\usepackage{pdfcolmk}
\usepackage{xcolor}
\usepackage{fancyhdr}
\usepackage{txfonts}
\usepackage{textcomp}

\usepackage{fontspec}
\defaultfontfeatures{Mapping=tex-text}
\usepackage{libertine}

\usepackage{fullpage} % to reduce the margins
\setlength{\hoffset}{-1in}
\setlength{\voffset}{-1in}
\setlength{\paperwidth}{155mm}
\setlength{\textwidth}{130mm}
\setlength{\oddsidemargin}{\paperwidth * \real{0.5} - \textwidth * \real{0.5}}
\setlength{\evensidemargin}{\paperwidth * \real{0.5} - \textwidth * \real{0.5}}
\setlength{\paperheight}{219.17mm}
\setlength{\topmargin}{10.0mm}
\setlength{\textheight}{179.17mm}
\setlength{\footskip}{10.0mm}
\setlength{\marginparwidth}{0mm}
\setlength{\marginparsep}{0mm}
\mag=1355 % Pour revenir en 210mm x 297mm

\pagestyle{fancy}
\headheight=20pt
\headsep=15pt

\usepackage{float}
 
\floatstyle{plain}
\newfloat{piece}{H}

\input ../commun/commun.tex

\begin{document}
\fontsize{12.5}{13.5}\selectfont
\parindent=0pt

\tolerance=200
\pretolerance=10
\hbadness=10000

% Debug : affichage des paramètres du layout
%\layout*

% Paramètrage des entêtes
\fancyhead{} % clear all header fields
\fancyhead[CE]{\bfseries \sc{office des ténèbres}}
\fancyhead[CO]{\bfseries \sc{samedi saint}}
\fancyfoot{} % clear all footer fields
\fancyfoot[LE,RO]{\thepage}
\fancyfoot[CO,CE]{\small{\textit{Schola Saint-Maur}}}
\renewcommand{\headrulewidth}{0.4pt}
\renewcommand{\footrulewidth}{0.4pt}

% Paramètrage de Gregorio
\setspaceafterinitial{9mm plus 0em minus 0em}
\setspacebeforeinitial{5mm plus 0em minus 0em}

\def\greinitialformat#1{%
{\fontsize{33}{33}\selectfont #1}%
}

\def\grebiginitialformat#1{%
{\fontsize{33}{33}\selectfont #1}%
}

% Quelques macros pour les partitions grégorien
\def\ssmAnnotation#1#2#3{%
\setspaceafterinitial {#1 plus 0em minus 0em}%
\setspacebeforeinitial{#1 plus 0em minus 0em}%
\gresetfirstannotation{\bf #2}%
\gresetsecondannotation{\bf #3}%
}

\def\ssmCommentary#1#2{%
\hfill\raisebox{#1}[0mm][0mm]{\color{red}{\small \emph{#2}}}\par
}

% Première page
%\begin{titlepage}

%\end{titlepage}

% ------------------- Premier nocturne
\newpage
\fancyhead[CO]{\bfseries \sc{samedi saint -- premier nocturne}}
\titrenocturne{Premier nocturne.}

\rubrique{On ne chante ni invitatoire, ni hymne, mais on commence 
directement par la première antienne, en se signant les lèvres.}

\rubrique{À la fin de chaque psaume, un des servants éteint l’un des 
quinze cierges du chandelier qui aura été placé en avant de l’autel.}

% Antienne 1
\vskip 12pt
\ssmAnnotation{6mm}{1. Ant.}{8. g}
\ssmCommentary{-1mm}{Cf. Ps. 4, 9}
\includescore{../800/ant_In_pace.tex}
\input{../commun/fr/ant_In_pace_fr}

% Psaume 4
\vskip 12pt
\input{../ps/fr/tit_ps4_fr}
\input{../ps/neo-vulgate/int_ps4_8}
\input{../ps/neo-vulgate/ps4_8}

% Antienne 2
\vskip 12pt
\vskip 12pt
\vskip 12pt
\ssmAnnotation{3mm}{2. Ant.}{4. e}
\ssmCommentary{0mm}{Cf. Ps. 14, 1}
\includescore{../800/ant_Habitabit.tex}
\input{../commun/fr/ant_Habitabit_fr}

% Psaume 14
\vskip 12pt
\vskip 12pt
\input{../ps/fr/tit_ps14_fr}
\input{../ps/neo-vulgate/int_ps14_4}
\kern -12pt
\input{../ps/neo-vulgate/ps14_4}

% Antienne 3
\vskip 12pt
\vskip 12pt
\vskip 12pt
\ssmAnnotation{4mm}{3. Ant.}{7. c}
\ssmCommentary{0mm}{Cf. Ps. 15, 9}
\includescore{../800/ant_Caro_mea.tex}
\kern -12pt
\input{../commun/fr/ant_Caro_mea_fr}

% Psaume 15
\input{../ps/fr/tit_ps15_fr}
\input{../ps/neo-vulgate/int_ps15_7}
\input{../ps/neo-vulgate/ps15_7}

\rubriquefinpsaumes

\ssmCommentary{0mm}{Cf. Ps. 21, 19}
\includescore{../commun/vers_In_pace_in_idipsum.tex}
\input{../commun/fr/vers_In_pace_in_idipsum_fr}

\rubriquefinverset

% Leçon I
\vskip 12pt
\titrelecture{Lecture I.}{Lam 3, 22-30}
\reflect{Lect_I}
\input{../lect/lt/lect_I_lt}

% Répons I
\vskip 12pt
\ssmAnnotation{5mm}{Rép. 1.}{2.}
\ssmCommentary{0mm}{Cf. Mt. 27, 62\,.\,67}
\includescore{../800/resp_Sepulto.tex}
\input{../commun/fr/resp_Sepulto_fr}

% Leçon II
\vskip 12pt
\titrelecture{Lecture II.}{Lam 4, 1-6}
\reflect{Lect_II}
\input{../lect/lt/lect_II_lt}

% Répons II
\vskip 12pt
\ssmAnnotation{6mm}{Rép. 2.}{5.}
\includescore{../800/resp_Ierusalem.tex}
\input{../commun/fr/resp_Ierusalem_fr}

% Leçon III
\vskip 12pt
\titrelecture{Lecture III.}{Lam 5, 1-11}
\reflect{Lect_III}
\input{../lect/lt/lect_III_lt}

% Répons III
\vskip 12pt
\ssmAnnotation{5mm}{Rép. 3.}{5.}
\ssmCommentary{0mm}{Cf. Mt. 26, 38\,.\,45}
\includescore{../800/resp_Plange.tex}
\input{../commun/fr/resp_Plange_fr}

\rubrique{On garde le silence un instant. Au signal du 
cérémoniaire, tous se lèvent, et on entonne la première 
antienne du deuxième nocturne.}

%\finnocturne

% ------------------- Deuxième nocturne
\newpage
\fancyhead[CO]{\bfseries \sc{samedi saint -- deuxième nocturne}}
\titrenocturne{Deuxième nocturne.}

% Antienne 1
\ssmAnnotation{4mm}{1. Ant.}{5. a}
\ssmCommentary{0mm}{Cf. Ps. 23, 9}
\includescore{../800/ant_Elevamini.tex}
\input{../commun/fr/ant_Elevamini_fr}

% Psaume 23
\input{../ps/fr/tit_ps23_fr}
\input{../ps/neo-vulgate/int_ps23_5}
\input{../ps/neo-vulgate/ps23_5}

% Antienne 2
\vskip 12pt
\ssmAnnotation{3mm}{2. Ant.}{4. e}
\ssmCommentary{0mm}{Cf. Ps. 26, 13}
\includescore{../800/ant_Credo_videre.tex}
\kern -12pt
\input{../commun/fr/ant_Credo_videre_fr}

% Psaume 26
\vskip 12pt
\input{../ps/fr/tit_ps26_fr}
\input{../ps/neo-vulgate/int_ps26_4}
\input{../ps/neo-vulgate/ps26_4}

% Antienne 3
\vskip 12pt
\ssmAnnotation{3mm}{3. Ant.}{8. g}
\ssmCommentary{0mm}{Cf. Ps. 29, 4}
\includescore{../800/ant_Domine.tex}
\kern -12pt
\input{../commun/fr/ant_Domine_abstraxisti_fr}

% Psaume 29
\input{../ps/fr/tit_ps29_fr}
\input{../ps/neo-vulgate/int_ps29_8}
\input{../ps/neo-vulgate/ps29_8}

\rubriquefinpsaumes

% Verset
\ssmCommentary{0mm}{Cf. Ps. 40, 11}
\includescore{../commun/vers_Tu_autem.tex}
\input{../commun/fr/vers_Tu_autem_fr}

\rubriquefinverset

% Leçon IV
\vskip 6pt
\titrelecture{Lecture IV.}{PG 43, 439\,.\,451\,.\,462-463}
\reflect{Lect_IV}
\input{../lect/LH/lt/lect_IV_lt}

% Répons IV
\vskip 12pt
\vskip 12pt
\ssmAnnotation{5mm}{Rép. 4.}{7.}
\includescore{../800/resp_Recessit.tex}
\input{../commun/fr/resp_Recessit_fr}

% Leçon V
\vskip 12pt
\vskip 12pt
\titrelecture{Lecture V.}{PG 43, 439\,.\,451\,.\,462-463}
\reflect{Lect_V}
\input{../lect/LH/lt/lect_V_lt}

% Répons V
\vskip 12pt
\vskip 12pt
\ssmAnnotation{4mm}{Rép. 5.}{8.}
\ssmCommentary{+1mm}{Cf. Mt. 26, 55}
\includescore{../800/resp_O_vos_omnes.tex}
\input{../commun/fr/resp_O_vos_omnes_fr}

% Leçon VI
\vskip 12pt
\titrelecture{Lecture VI.}{PG 43, 439\,.\,451\,.\,462-463}
\reflect{Lect_VI}
\input{../lect/LH/lt/lect_VI_lt}

% Répons VI
\vskip 12pt
\ssmAnnotation{5mm}{Rép. 6.}{4.}
\ssmCommentary{-1mm}{Cf. Is. 57, 1; 53, 7-8}
\includescore{../800/resp_Ecce.tex}
\input{../commun/fr/resp_Ecce_fr}

\rubrique{On garde le silence un instant. Au signal du cérémoniaire, 
tous se lèvent, et on entonne la première antienne du troisième nocturne.}

%\finnocturne

% ------------------- Troisième nocturne
\newpage
\fancyhead[CO]{\bfseries \sc{samedi saint -- troisième nocturne}}
\titrenocturne{Troisième nocturne.}

% Antienne 1
\ssmAnnotation{5mm}{1. Ant.}{8. g}
\ssmCommentary{-1mm}{Cf. Ps. 53, 6}
\includescore{../800/ant_Deus_adiuvat_me.tex}
\input{../commun/fr/ant_Deus_adiuvat_me_fr}

% Psaume 53
\input{../ps/fr/tit_ps53_fr}
\input{../ps/neo-vulgate/int_ps53_8}
\input{../ps/neo-vulgate/ps53_8}

% Antienne 2
\vskip 12pt
\ssmAnnotation{6mm}{2. Ant.}{7. a}
\ssmCommentary{1mm}{Cf. Ps. 75, 3}
\includescore{../800/ant_In_pace_factus_est.tex}
\input{../commun/fr/ant_In_pace_factus_est_fr}

% Psaume 75
\input{../ps/fr/tit_ps75_fr}
\input{../ps/neo-vulgate/int_ps75_7}
\input{../ps/neo-vulgate/ps75_7}

% Antienne 3
\vskip 12pt
\vskip 12pt
\vskip 12pt
\vskip 12pt
\ssmAnnotation{5mm}{3. Ant.}{4. d}
\ssmCommentary{0mm}{Cf. Ps. 87, 5-6}
\includescore{../800/ant_Factus_sum.tex}
\input{../commun/fr/ant_Factus_sum_fr}

% Psaume 87
\input{../ps/fr/tit_ps87_fr}
\input{../ps/neo-vulgate/int_ps87_4}
\input{../ps/neo-vulgate/ps87_4}

\rubriquefinpsaumes

% Verset
\ssmCommentary{0mm}{Cf. Ps. 75, 3}
\includescore{../commun/vers_In_pace_factus_est.tex}
\input{../commun/fr/vers_In_pace_factus_est_fr}

\rubriquefinverset

% Leçon VII
\titrelecture{Lecture VII.}{He 4, 1-5}
\reflect{Lect_VII}
\input{../lect/LH/lt/lect_VII_lt}

% Répons VII
\vskip 12pt
\ssmAnnotation{4mm}{Rép. 7.}{4.}
\ssmCommentary{-1mm}{Cf. Ps. 87, 5\,.\,6\,.\,7}
\includescore{../800/resp_Aestimatus_sum.tex}
\input{../commun/fr/resp_Aestimatus_sum_fr}

% Leçon VIII
\vskip 12pt
\titrelecture{Lecture VIII.}{He 4, 6-10}
\reflect{Lect_VIII}
\input{../lect/LH/lt/lect_VIII_lt}

% Répons VIII
\vskip 12pt
\ssmAnnotation{3mm}{Rép. 8.}{8.}
\ssmCommentary{0mm}{Cf. Ps. 2, 2\,.\,1}
\includescore{../800/resp_Astiterunt.tex}
\input{../commun/fr/resp_Astiterunt_fr}

% Leçon IX
\vskip 12pt
\titrelecture{Lecture IX.}{He 4, 11-13}
\reflect{Lect_IX}
\input{../lect/LH/lt/lect_IX_lt}

% Répons IX
\vskip 12pt
\ssmAnnotation{5mm}{Rép. 9.}{4.}
\ssmCommentary{-1mm}{Cf. Is 53, 7\,.\,41}
\includescore{../800/resp_Sicut_ovis.tex}
\input{../commun/fr/resp_Sicut_ovis_fr}

\rubrique{On garde le silence un instant. Au signal du cérémoniaire, 
tous se lèvent et se tournent vers l’autel.  L’officiant  entonne la 
première antienne des Laudes, et tous se signent.}

%\finnocturne

% ------------------- Laudes
\newpage
\fancyhead[CO]{\bfseries \sc{samedi saint -- laudes}}
\titrelaudes{Laudes}

% Antienne 1
\ssmAnnotation{5mm}{1. Ant.}{2* d}
\ssmCommentary{0mm}{Cf. Rom 8, 32}
\includescore{../AM2005/ant_O_mors.tex}
\input{../commun/fr/ant_O_mors_fr}

% Psaume 50
\input{../ps/fr/tit_ps50_fr}
\input{../ps/neo-vulgate/int_ps50_2_etoile}
\input{../ps/neo-vulgate/ps50_2_etoile}

% Antienne 2
\vskip 12pt
\ssmAnnotation{5mm}{2. Ant.}{2* a}
\ssmCommentary{1mm}{Cf. Za 12, 10}
\includescore{../AM2005/ant_Plangent_eum.tex}
\input{../commun/fr/ant_Plangent_eum_fr}

% Psaume 91
\input{../ps/fr/tit_ps91_fr}
\input{../ps/neo-vulgate/int_ps91_2_etoile}
\input{../ps/neo-vulgate/ps91_2_etoile}

% Antienne 3
\vskip 12pt
\ssmAnnotation{4mm}{3. Ant.}{7. b}
\ssmCommentary{0mm}{Cf. Lm 1, 19}
\includescore{../AM2005/ant_Attendite.tex}
\input{../commun/fr/ant_Attendite_fr}

% Psaume 63
\input{../ps/fr/tit_ps63_fr}
\input{../ps/neo-vulgate/int_ps63_7}
\input{../ps/neo-vulgate/ps63_7}

% Antienne 4
\vskip 12pt
\ssmAnnotation{5mm}{4. Ant.}{2. d}
%\ssmCommentary{-1mm}{Cf. Lm 1, 19}
\includescore{../AM2005/ant_A_porta_inferi.tex}
\kern -12pt
\input{../commun/fr/ant_A_porta_inferi_fr}

% Psaume AT23
\input{../ps/fr/tit_at23_fr}
\input{../ps/neo-vulgate/int_at23_2}
\kern -12pt
\input{../ps/neo-vulgate/at23_2}

% Antienne 5
\vskip 12pt
\vskip 12pt
\ssmAnnotation{4mm}{5. Ant.}{8. c}
\ssmCommentary{+1mm}{Cf. Lm 1, 12}
\includescore{../AM2005/ant_O_vos_omnes.tex}
\kern -12pt
\input{../commun/fr/ant_O_vos_omnes_fr}

% Psaume 150
\input{../ps/fr/tit_ps148_fr}
\input{../ps/neo-vulgate/int_ps148_8}
\kern -12pt
\input{../ps/neo-vulgate/ps148_8}

\vskip 12pt
\input{../ps/fr/tit_ps149_fr}
\input{../ps/neo-vulgate/int_ps149_8}
%\kern -12pt
\input{../ps/neo-vulgate/ps149_8}

\vskip 12pt
\input{../ps/fr/tit_ps150_fr}
\input{../ps/neo-vulgate/int_ps150_8}
\kern -12pt
\input{../ps/neo-vulgate/ps150_8}

\rubrique{On ne dit pas de Lecture Brève, mais au signal du
cérémoniaire, tous se lèvent et, tournés vers l’autel, répondent 
au verset proclamé par un chantre. L’officiant entonne alors 
l’antienne du \rubriqueBlack{Benedíctus}.}

% Verset
\ssmCommentary{-1mm}{Cf. Ps. 15, 9\,.\,10}
\includescore{../commun/vers_Caro_mea.tex}
\input{../commun/fr/vers_Caro_mea_fr}

% Antienne du Benedictus
\vskip 10pt
\ssmAnnotation{6mm}{A Bened.}{1. g}
\includescore{../AM2005/ant_Mulieres.tex}
\input{../commun/fr/ant_Mulieres_fr}

% Benedictus
\begin{center}
{\bf {\sc {Benedictus}}}
\end{center}
\kern -12pt
\ssmCommentary{-1mm}{Cf. Lc 1, 68-79}
\input{../ps/neo-vulgate/int_benedictus_1}
\input{../ps/neo-vulgate/benedictus_1}

\rubrique{%
Les cierges du chandelier triangulaire ont été successivement 
éteints. Un seul, placé au sommet du chandelier, est resté 
allumé.  Pendant le \rubriqueBlack{Benedíctus}, on éteint les 
six cierges qui brûlent sur l’autel, de chaque côté alternativement, 
de manière que tous soient éteints au dernier verset.  On éteint 
aussi toutes les lumières de l’église.}

\rubrique{%
Après la reprise de l’antienne \rubriqueBlack{Mulíeres},
tous se mettent à genoux et chantent le \rubriqueBlack{Christus}, pendant 
lequel un des servants retire le cierge resté allumé, et va le dissimuler 
derrière l’autel.}

\ssmCommentary{1mm}{Cf. Phil 2,8-9}
\ssmAnnotation{2mm}{}{5.}
\includescore{../800/ant_Christus.tex}
\input{../commun/fr/ant_Christus_fr}

\rubrique{%
On garde le silence le temps d’un \rubriqueBlack{Notre Père}.
L’officiant lit alors l’oraison sans dire \rubriqueBlack{Orémus},
sur un ton assez grave, en descendant d’un ton sur la dernière syllabe 
et sans conclusion.}

% Oraison
\begin{center}
\textbf{Oraison.}
\end{center}
\input{../commun/oratio}

\rubrique{L’oraison terminée, on frappe avec bruit sur les stalles, 
jusqu’à ce que le cierge resté allumé soit déposé au sommet du chandelier. 
Au signal du cérémoniaire, tous se lèvent alors et se retirent en silence.}

\finoffice

% ------------------- Annexes
\newpage
\fancyhead[CO]{\bfseries \sc{samedi saint -- annexes}}
\begin{appendices}

\label{Lect_I}
\titrelecture{Lecture I.}{}
\includescore{../lect/lect_I.tex}

\newpage
\label{Lect_II}
\titrelecture{Lecture II.}{}
\includescore{../lect/lect_II.tex}

\vskip 36pt
\label{Lect_III}
\titrelecture{Lecture III.}{}
\includescore{../lect/lect_III.tex}

\vskip 36pt
\label{Lect_IV}
\titrelecture{Lecture IV.}{}
\includescore{../lect/LH/lect_IV.tex}

\vskip 36pt
\label{Lect_V}
\titrelecture{Lecture V.}{}
\includescore{../lect/LH/lect_V.tex}

\vskip 36pt
\label{Lect_VI}
\titrelecture{Lecture VI.}{}
\includescore{../lect/LH/lect_VI.tex}

\newpage
\label{Lect_VII}
\titrelecture{Lecture VII.}{}
\includescore{../lect/LH/lect_VII.tex}

\vskip 36pt
\label{Lect_VIII}
\titrelecture{Lecture VIII.}{}
\includescore{../lect/LH/lect_VIII.tex}

\newpage
\label{Lect_IX}
\titrelecture{Lecture IX.}{}
\includescore{../lect/LH/lect_IX.tex}

\end{appendices}

% Index

%\newpage

%\tableofcontents

\end{document}
