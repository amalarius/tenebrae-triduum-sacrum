\documentclass[11pt,twoside]{article} % use larger type; default would be 10pt

% usual packages loading:
\usepackage{luatextra}
\usepackage{graphicx} % support the \includegraphics command and options
\usepackage{geometry} % See geometry.pdf to learn the layout options. There are lots.
\geometry{a4paper} % or letterpaper (US) or a5paper or....
\usepackage{gregoriotex} % for gregorio score inclusion
\usepackage{fullpage} % to reduce the margins
\usepackage[toc,page]{appendix}

% choose the language of the document here
\usepackage[latin,frenchb]{babel}

% use the two following package for using normal TeX fonts
\usepackage[T1]{fontenc}
\usepackage[utf8]{luainputenc}

% If you use usual TeX fonts, here is a starting point:
\usepackage{times}

\usepackage{layout}

\usepackage{parallel}
\usepackage{color}
\usepackage{pdfcolmk}
\usepackage{xcolor}
\usepackage{fancyhdr}
\usepackage{txfonts}
\usepackage{textcomp}

\pagestyle{fancy}
%\topmargin=10pt
\headheight=20pt
\headsep=15pt

\usepackage{float}
 
\floatstyle{plain}
\newfloat{piece}{H}

\input commun.tex

\begin{document}
% Debug : affichage des paramètres du layout
\layout*

% Paramètrage des entètes
\fancyhead{} % clear all header fields
\fancyhead[CE]{\bfseries Office des Ténèbres}
\fancyhead[CO]{\bfseries Samedi Saint}
\fancyfoot{} % clear all footer fields
\fancyfoot[LE,RO]{\thepage}
\fancyfoot[CO,CE]{Schola Saint-Maur}
\renewcommand{\headrulewidth}{0.4pt}
\renewcommand{\footrulewidth}{0.4pt}

% Paramètrage de Gregorio
\setspaceafterinitial{9mm plus 0em minus 0em}
\setspacebeforeinitial{5mm plus 0em minus 0em}

\def\greinitialformat#1{%
{\fontsize{55}{50}\selectfont #1}%
}

\def\grebiginitialformat#1{%
{\fontsize{73}{73}\selectfont #1}%
}

\setgrefactor{22}
\def\gretextformat#1{%
{\fontsize{14}{14}\selectfont #1\relax}%
}

% Premier nocturne

\newpage

\titreantienne{Antienne 1.}

\gresetfirstlineaboveinitial{\scriptsize \textsc{\textbf{1 \Abar. VIII g}}}{\scriptsize \textsc{\textbf{1 \Abar. VIII g}}}
\commentary{\color{red}{\emph{Cf. Ps 4,9}}}
\nopagebreak
\includescore{ant_In_pace.tex}

\input{./fr/ant_In_pace_fr}

% Psaume 4
%\input{ps4}

% Antienne 2

\titreantienne{Antienne 2.}

\gresetfirstlineaboveinitial{\scriptsize \textsc{\textbf{2 \Abar. IV e}}}{\scriptsize \textsc{\textbf{2 \Abar. IV e}}}
\commentary{\color{red}{\emph{Cf. Ps 14,1}}}
\nopagebreak
\includescore{ant_Habitabit.tex}

\input{./fr/ant_Habitabit_fr}

% Psaume 14

% Antienne 3

\titreantienne{Antienne 3.}

\gresetfirstlineaboveinitial{\scriptsize \textsc{\textbf{3 \Abar. VII c}}}{\scriptsize \textsc{\textbf{3 \Abar. VII c}}}
\commentary{\color{red}{\emph{Cf. Ps 15,9}}}
\nopagebreak
\includescore{ant_Caro_mea.tex}

\input{./fr/ant_Caro_mea_fr}

% Psaume 15

\commentary{\color{red}{\emph{Cf. Ps 21,19}}}
\nopagebreak
\includescore{vers_In_pace_in_idipsum.tex}

\input{./fr/vers_In_pace_in_idipsum_fr}

% Leçon I

\titrelecture{Lecture I.}

% Répons I

\titrerepons{Répons 1.}

\commentary{\color{red}{\emph{Cf. Mt 27,62.67}}}
\nopagebreak
\includescore{resp_Sepulto.tex}

\input{./fr/resp_Sepulto_fr}

% Leçon II

\titrelecture{Lecture II.}

% Répons II

\titrerepons{Répons 2.}

\includescore{resp_Ierusalem.tex}

\input{./fr/resp_Ierusalem_fr}

% Leçon III
\titrelecture{Lecture III.}

% Répons III

\titrerepons{Répons 2.}

\commentary{\color{red}{\emph{Cf. Mt 26,38.45}}}
\nopagebreak
\includescore{resp_Plange.tex}

\input{./fr/resp_Plange_fr}

% Deuxième nocturne

\newpage

% Antienne 1

\titreantienne{Antienne 1.}

\gresetfirstlineaboveinitial{\scriptsize \textsc{\textbf{1 \Abar. V a}}}{\scriptsize \textsc{\textbf{1 \Abar. V a}}}
\commentary{\color{red}{\emph{Cf. Ps. 23,9}}}
\nopagebreak
\includescore{ant_Elevamini.tex}

\input{./fr/ant_Elevamini_fr}

% Psaume 23

% Antienne 2

\titreantienne{Antienne 2.}

\gresetfirstlineaboveinitial{\scriptsize \textsc{\textbf{2 \Abar. IV e}}}{\scriptsize \textsc{\textbf{2 \Abar. IV e}}}
\commentary{\color{red}{\emph{Cf. Ps 26, 13}}}
\nopagebreak
\includescore{ant_Credo_videre.tex}

\input{./fr/ant_Credo_videre_fr}

% Psaume 26

% Antienne 2

\titreantienne{Antienne 3.}

\gresetfirstlineaboveinitial{\scriptsize \textsc{\textbf{3 \Abar. VIII g}}}{\scriptsize \textsc{\textbf{3 \Abar. VIII g}}}
\commentary{\color{red}{\emph{Cf. Ps. 29,4}}}
\nopagebreak
\includescore{ant_Domine_abstraxisti.tex}

\input{./fr/ant_Domine_abstraxisti_fr}

% Psaume 29

% Verset

\commentary{\color{red}{\emph{Cf. Ps 40,11}}}
\nopagebreak
\includescore{vers_Tu_autem.tex}

\input{./fr/vers_Tu_autem_fr}

% Leçon IV
\titrelecture{Lecture IV.}

% Répons IV

\titrerepons{Répons 4.}

\includescore{resp_Recessit.tex}

\input{./fr/resp_Recessit_fr}

% Leçon V
\titrelecture{Lecture V.}

% Répons V

\titrerepons{Répons 5.}

\commentary{\color{red}{\emph{Cf. Mt 26,55}}}
\nopagebreak
\includescore{resp_O_vos_omnes.tex}

\input{./fr/resp_O_vos_omnes_fr}

% Leçon VI
\titrelecture{Lecture VI.}

% Répons VI

\titrerepons{Répons 6.}

\commentary{\color{red}{\emph{Cf. Is 57,1; 53,7-8}}}
\nopagebreak
\includescore{resp_Ecce.tex}

\input{./fr/resp_Ecce_fr}

% Troisième nocturne

\newpage

% Antienne 1

\titreantienne{Antienne 1.}

\gresetfirstlineaboveinitial{\scriptsize \textsc{\textbf{1 \Abar. VIII g}}}{\scriptsize \textsc{\textbf{1 \Abar. VIII g}}}
\commentary{\color{red}{\emph{Cf. Ps 53, 6}}}
\nopagebreak
\includescore{ant_Deus_adiuvat_me.tex}

\input{./fr/ant_Deus_adiuvat_me_fr}

% Psaume 53

% Antienne 2

\titreantienne{Antienne 2.}

\gresetfirstlineaboveinitial{\scriptsize \textsc{\textbf{2 \Abar. VII a}}}{\scriptsize \textsc{\textbf{2 \Abar. VII a}}}
\commentary{\color{red}{\emph{Cf. Ps 75,3}}}
\nopagebreak
\includescore{ant_In_pace_factus_est.tex}

\input{./fr/ant_In_pace_factus_est_fr}

% Psaume 75

% Antienne 3

\titreantienne{Antienne 3.}

\gresetfirstlineaboveinitial{\scriptsize \textsc{\textbf{3 \Abar. IV d}}}{\scriptsize \textsc{\textbf{3 \Abar. IV d}}}
\commentary{\color{red}{\emph{Cf. Ps 87,5-6}}}
\nopagebreak
\includescore{ant_Factus_sum.tex}

\input{./fr/ant_Factus_sum_fr}

% Psaume 87

\commentary{\color{red}{\emph{Cf. Ps 75,3}}}
\nopagebreak
\includescore{vers_In_pace_factus_est.tex}

\input{./fr/vers_In_pace_factus_est_fr}

% Leçon VII
\titrelecture{Lecture VII.}

% Répons VII
\titrerepons{Répons 7.}

\commentary{\color{red}{\emph{Cf. Ps 87, 5.6.7}}}
\nopagebreak
\includescore{resp_Aestimatus_sum.tex}

\input{./fr/resp_Aestimatus_sum_fr}

% Leçon VIII
\titrelecture{Lecture VIII.}

% Répons VIII
\titrerepons{Répons 8.}

\commentary{\color{red}{\emph{Cf. Ps 2,2.1}}}
\nopagebreak
\includescore{resp_Astiterunt.tex}

\input{./fr/resp_Astiterunt_fr}

% Leçon IX
\titrelecture{Lecture IX.}

% Répons IX
\titrerepons{Répons 9.}

\commentary{\color{red}{\emph{Cf. Is 53,7.41}}}
\nopagebreak
\includescore{resp_Sicut_ovis.tex}

\input{./fr/resp_Sicut_ovis_fr}

% Laudes

\newpage

% Antienne 1

\titreantienne{Antienne 1.}

\gresetfirstlineaboveinitial{\scriptsize \textsc{\textbf{1 \Abar. IV c}}}{\scriptsize \textsc{\textbf{1 \Abar. IV c}}}
\commentary{\color{red}{\emph{Cf. Rom 8,32}}}
\nopagebreak
\includescore{ant_O_mors.tex}

\input{./fr/ant_O_mors_fr}

% Psaume 50

% Antienne 2

\titreantienne{Antienne 2.}

\gresetfirstlineaboveinitial{\scriptsize \textsc{\textbf{2 \Abar. II* a}}}{\scriptsize \textsc{\textbf{2 \Abar. II* a}}}
\commentary{\color{red}{\emph{Cf. Za 12,10}}}
\nopagebreak
\includescore{ant_Plangent_eum.tex}

\input{./fr/ant_Plangent_eum_fr}

% Psaume 91

% Antienne 3

\titreantienne{Antienne 3.}

\gresetfirstlineaboveinitial{\scriptsize \textsc{\textbf{3 \Abar. VII b}}}{\scriptsize \textsc{\textbf{3 \Abar. VII b}}}
\commentary{\color{red}{\emph{Cf. Lm 1, 18}}}
\nopagebreak
\includescore{ant_Attendite.tex}

\input{./fr/ant_Attendite_fr}

% Psaume 63

% Antienne 4

\titreantienne{Antienne 4.}

\gresetfirstlineaboveinitial{\scriptsize \textsc{\textbf{4 \Abar. II d}}}{\scriptsize \textsc{\textbf{4 \Abar. II d}}}
\includescore{ant_A_porta_inferi.tex}

\input{./fr/ant_A_porta_inferi_fr}

%\input{../ps/neo-vulgate/ps63}

% Psaume AT23

% Antienne 5

\titreantienne{Antienne 5.}

\gresetfirstlineaboveinitial{\scriptsize \textsc{\textbf{5 \Abar. VIII c}}}{\scriptsize \textsc{\textbf{5 \Abar. VIII c}}}
\commentary{\color{red}{\emph{Cf. Lm 1,12}}}
\nopagebreak
\includescore{ant_O_vos_omnes.tex}

\input{./fr/ant_O_vos_omnes_fr}

% Psaume 150

% Verset
\commentary{\color{red}{\normalsize \emph{Cf. Ps 15, 9.10}}}
\nopagebreak
\includescore{vers_Caro_mea.tex}

\input{./fr/vers_Caro_mea_fr}

% Antienne du Benedictus

\titreantienne{Antienne du Benedictus.}

\gresetfirstlineaboveinitial{\scriptsize \textsc{\textbf{\Abar. I g}}}{\scriptsize \textsc{\textbf{\Abar. I g}}}
\includescore{ant_Mulieres.tex}

\input{./fr/ant_Mulieres_fr}

% Benedictus

\gresetfirstlineaboveinitial{\scriptsize \textsc{\textbf{\Abar.}}}{\scriptsize \textsc{\textbf{\Abar.}}}
\commentary{\color{red}{\emph{Cf. Phil 2,8-9}}}
\nopagebreak
\includescore{ant_Christus.tex}

\input{./fr/ant_Christus_fr}

% Oraison

% Annexes
\newpage

\begin{appendices}

\label{Lect_I}
\titrelecture{Lecture I.}
\includescore{../lect/lect_I.tex}

\newpage
\label{Lect_II}
\titrelecture{Lecture II.}
\includescore{../lect/lect_II.tex}

\newpage
\label{Lect_III}
\titrelecture{Lecture III.}
\includescore{../lect/lect_III.tex}

\end{appendices}

% Index

\newpage

\tableofcontents

\end{document}
